\documentclass{article}
\title{Golf-Sport:  Managing Operations \\ A Case Study in Optimization}
\date{May 8, 2018}
\author{John D. Bulger \& Matthew Hess, Valparaiso University}
\usepackage[margin=1in]{geometry}
\begin{document}
	\pagenumbering{gobble}
	\maketitle
	\newpage
	\pagenumbering{arabic}
\section{Introduction}
This problem examines a manufacturing optimization case........
\section{Background and Prior Work}
Here is a good spot to insertour necessary references.  We can evaluate solutions of similar problems
\section{Description of Problem}
Golf-Sport is a golfclub manufacturing company with three plants in Arizona.  Each plant has an attached retail store and is treated as its
only supplier.  Each location has predicted minimum and maximum sales of each item type per period.  The goal of this optimization is to maximize profit over a two-month period by determining the optimal number of each product to make
at each location.  In this problem, two types of constraints exist:  local \& global.  Local constraints are unique to each plant/store
 combination, while the global constraints refer to company-wide resources.
\section{Mathematical Formulation \& Model}
This problem is set up as a mixed integer programming problem, with a single objective function to be maximized (representing profit).  
The decision variable are set to be the production of each product item, indexed by location and period.  In determining the maximum profit given the constraints, 
the solution will also yield the optimal production plan for Golf-Sport.
\subsection{Assumptions}
Two main assumptions were made in order to create an effective model for this problem.  First, this model represents the company's production as if it were not a 
going-concern.  The model is constructed to optimize profit over two months, thereby not building up any additional inventory for carryover into 
a third month.  By default, the model will not have any inventory left at the end of the second month.  If further information were to arise along with 
a modeling need for a third month, this could be accomplished by expanding the current model. \par
The second assumption used is the carryover of graphite stock between months.  Advertising and graphite allotments are companywide in this problem, and 
it is clearly stated that the amount of money earmarked for advertising each month will not carryover into the next month.  Graphite seems to have no such restriction, so 
it appears reasonable to incorporate the possibility of carryover of graphite stock.
\subsection{Indices}
\begin{tabular}{ c c }
\textbf{Index} & \textbf{Location} \\
\textit{C} & Chandler \\
\textit{G} & Glendale \\
\textit{T} & Tucson \\
\textbf{Index} & \textbf{Period Description} \\
\textit{1} & Produced \& Sold in Period 1 \\
\textit{2} & Produced \& Sold in Period 2 \\
\textit{3} & Produced in Period 1 \& Sold in Period 2 \\
\end{tabular}
\subsection{Variables}
\begin{tabular}{ c c }
\textbf{Variable} & \textbf{Product} \\
\textit{s} & Steel Shafts \\
\textit{g} & Graphite Shafts \\
\textit{i} & Forged Iron Heads \\
\textit{w} & Metal Wood Heads \\
\textit{h} & Titanium Insert Wood Heads \\
\textit{v} & Set, Steel Shafts, Metal Heads \\
\textit{x} & Set, Steel Shafts, Insert Heads \\
\textit{y} & Set, Graphite Shafts, Metal Heads \\
\textit{z} & Set, Graphite Shafts, Insert Heads \\
\end{tabular}
\subsection{Objective}
Here we put in our objective function and explain the goal behind it
\subsection{Constraints}
Here we list all constraints in with an explanation surrounding each
\subsection{Heuristics}
Just kidding, this problem is not big enough for any heuristics
\section{Implementation}
\subsection{Hardware \& Software}
Problem was solved using MATLAB R2017B  It was run on a computer.....
\subsection{Coding}
%Maybe insert a page of pseudocode here???????
Code
\subsection{Algorithm}
Run as an IP using optimproblem, which is essentially an interior points method of solution.  Will run an LP to examine solutions
\section{Solution}
Explain results, compare to linear relaxation of the problem
\par
maybe insert a section with a graph or chart to visualize production
\section{Further Analysis}
Conduct range sensitivity, examine extra problems in case study
\section{Conclusions \& Implications}
\subsection{Mathematical Approach}
Evaluate our approach, suggest changes in future.  EFFICIENCY
\subsection{Company Approach}
Discuss optimized solution, range sensitivity, business suggestions
	\newpage
	\pagenumbering{gobble}
	\begin{thebibliography}{9}
\bibitem{bulger88}

\end{thebibliography}
\end{document}
