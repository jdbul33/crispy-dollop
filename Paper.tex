\documentclass{article}
\title{Golf-Sport:  Managing Operations \\ A Case Study in Optimization}
\date{May 8, 2018}
\author{John D. Bulger \& Matthew Hess, Valparaiso University}
\usepackage[margin=1in]{geometry}
\begin{document}
	\pagenumbering{gobble}
	\maketitle
	\newpage
	\pagenumbering{arabic}
\section{Introduction}
This problem examines a manufacturing optimization case........
\section{Background and Prior Work}
Here is a good spot to insertour necessary references.  We can evaluate solutions of similar problems
\section{Description of Problem}
Golf-Sport is a golfclub manufacturing company with three plants in Arizona.  Each plant has an attached retail store and is treated as its
only supplier.  Each location has predicted minimum and maximum sales of each item type per period.  The goal of this optimization is to maximize profit over a two-month period by determining the optimal number of each product to make
at each location.  In this problem, two types of constraints exist:  local \& global.  Local constraints are unique to each plant/store
 combination, while the global constraints refer to company-wide resources.
\section{Mathematical Formulation \& Model}
This problem is set up as a mixed integer programming problem, with a single objective function to be maximized (representing profit).  
The decision variable are set to be the production of each product item, indexed by location and period.  In determining the maximum profit given the constraints, 
the solution will also yield the optimal production plan for Golf-Sport.
\subsection{Assumptions}
Two main assumptions were made in order to create an effective model for this problem.  First, this model represents the company's production as if it were not a 
going-concern.  The model is constructed to optimize profit over two months, thereby not building up any additional inventory for carryover into 
a third month.  By default, the model will not have any inventory left at the end of the second month.  If further information were to arise along with 
a modeling need for a third month, this could be accomplished by expanding the current model. \par
The second assumption used is the carryover of graphite stock between months.  Advertising and graphite allotments are companywide in this problem, and 
it is clearly stated that the amount of money earmarked for advertising each month will not carryover into the next month.  Graphite seems to have no such restriction, so 
it appears reasonable to incorporate the possibility of carryover of graphite stock.
\subsection{Variables \& Indices}
\begin{tabular}{ c c }
\textbf{Variable} & \textbf{Description} \\
\textit{x} & Production of Product \textit{i} in Period \textit{j} \\
\textit{e} & Graphite Stock Carryover \\
\end{tabular}

\begin{tabular}{ c c }
\textbf{Superscript} & \textbf{Location} \\
\textit{C} & Chandler \\
\textit{G} & Glendale \\
\textit{T} & Tucson \\
\end{tabular}

\begin{tabular}{ c c }
\textbf{Index \textit{i}} & \textbf{Product} \\
\textit{s} & Steel Shafts \\
\textit{g} & Graphite Shafts \\
\textit{r} & Forged Iron Heads \\
\textit{w} & Metal Wood Heads \\
\textit{h} & Titanium Insert Wood Heads \\
\textit{v} & Set, Steel Shafts, Metal Heads \\
\textit{u} & Set, Steel Shafts, Insert Heads \\
\textit{y} & Set, Graphite Shafts, Metal Heads \\
\textit{z} & Set, Graphite Shafts, Insert Heads \\
\end{tabular}
\begin{tabular}{ c c }
\textbf{Index \textit{j}} & \textbf{Period Description} \\
\textit{1} & Produced \& Sold in Period 1 \\
\textit{2} & Produced \& Sold in Period 2 \\
\textit{3} & Produced in Period 1 \& Sold in Period 2 \\
\end{tabular}

\subsection{Objective}
The objective of this model is to maximize profit, represented as the difference between sales revenue and cost.  For the purposes of this function let \textit{R, C \& I} represent revenue, cost, and inventory cost, respectively.  The function can be expressed as:
%align functions to the left

$$f(x) = \sum_{j=1}^{3}Rx_{i} - \sum_{j=1}^{3}Cx_{i} - \sum Ix_{i,3} $$

The total revenue from all products produced and sold, less the cost of manufacture and inventory, is represented in this function.  While designed for a two-period optimization, the base function can be easily expanded to accomodate for more periods of production and sales.
%superscript for plant, sub for product and period??
\subsection{Constraints}
In construction of this model, the following constraints were constructed within the parameters set forth by the problem:
\begin{equation}
\sum c_{i}x_{i,1} + \sum c_{i}x_{i,3} \leq L_{i}
\end{equation}
\begin{equation}
\sum c_{i}x_{i,2} \leq L_{i}
\end{equation}
\begin{equation}
\sum c_{i}x_{i,1} + \sum c_{i}x_{i,3} \leq P_{i}
\end{equation}
\begin{equation}
\sum c_{i}x_{i,2} \leq P_{i}
\end{equation}
\begin{equation}
\sum c_{i}x_{i,1} + \sum c_{i}x_{i,3} \leq A
\end{equation}
\begin{equation}
\sum c_{i}x_{i,2} \leq A
\end{equation}
\begin{equation}
\sum c_{i}x_{i,1} + \sum c_{i}x_{i,3} + e \leq G_{1} for i = g, y, z
\end{equation}
\begin{equation}
\sum c_{i}x_{i,2} \leq G_{2} + e  for i = g, y, z
\end{equation}
\begin{equation}
QQQQ \leq x_{i,1} \leq S_{1} %What is a good variable for the constants for sales bounds
\end{equation}
\begin{equation}
QQQQ \leq x_{i,2} + x_{i,3} \leq S_{2}
\end{equation}
\begin{equation}
x_{i,j} \geq 0
\end{equation}


\section{Implementation}
\subsection{Hardware \& Software}
The problem was coded and solved in MATLAB R2017b on a personal computer.  The code does not appear to be backward-compatible due to the use of the \textit{optimproblem} implementation.  The small scale of this optimization does not necessitate a computer beyond average specifications.
\subsection{Coding}
%Maybe insert a page of pseudocode here???????
Code
\subsection{Algorithm}
This problem was solved using the \textit{intlinprog} function in MATLAB.  This function conducts a linear relaxation on the problem, in order to set the upper bound.  Then, the algorithm iterates through a branch-and-cut approach to solve for the optimal integer solution.  By default, the function 
will iterate through 1,000,000 iterations prior to terminating.  Additionally, the function applies basic heuristics in the form of Gomory cuts.  Gomory cuts 
\section{Solution}
Explain results, compare to linear relaxation of the problem
\par
maybe insert a section with a graph or chart to visualize production
\section{Further Analysis}
Conduct range sensitivity, examine extra problems in case study
\section{Conclusions \& Implications}
\subsection{Mathematical Approach}
Evaluate our approach, suggest changes in future.  EFFICIENCY
\subsection{Company Approach}
Discuss optimized solution, range sensitivity, business suggestions
	\newpage
	\pagenumbering{gobble}
	\begin{thebibliography}{9}
\bibitem{bulger88}

\end{thebibliography}
\end{document}
