\documentclass{article}
\title{Golf-Sport:  Managing Operations \\ A Case Study in Optimization}
\date{May 8, 2018}
\author{John D. Bulger \& Matthew T. Hess, Valparaiso University}
\usepackage[margin=1in]{geometry}
\begin{document}
	\pagenumbering{gobble}
	\maketitle
	\newpage
	\pagenumbering{arabic}
\section{Introduction}
This problem examines a manufacturing optimization case, in which a company is looking to create an optimal two-month business plan.  The case study seeks to develop a production plan that will optimize profit over this production period.  
The two-month period is set, with no information given as to the going concern of this company.  This two-month optimization can serve as a baseline, and could later be expanded to more periods if desired.  This particular problem appears to lend itself to integer programming as the ideal optimization method.  The problem will be coded and solved using MATLAB's optimization toolbox functionality.
\subsection{Background \& Prior Work}
Optimizing profit through production in processes is studied within the realm of the operations management.  As the global market continues to grow and expand, businesses are focusing on remaining competitive.  In a survey of 
optimization techniques being used within operations management, the authors broken the process into three distinct parts.  These can be identified as input management, facilities management, and output management.  Stated more simply, these correspond to 
scheduling, production, and vehicle routing, respectively.  This Golf-Sport problem falls squarely within the facilities management/production sector.  The authors discovered that for large-scale production optimizations, heuristics were necessarily utilized.  
Additionally, the most common solution approaches utilize branch-and-cut or branch-and-price \cite{survey}.
\par
Several approaches for operations management use genetic algorithms, especially when focusing on lot sizing within material requirements, aggregate and assortment planning, and facility layout problems \cite{gen}.  
Additionally, particle swarm optimization methods have been successfully implemented into operations research problems.  Nearchou expands on this by stating that ``A possible reason for this absence is that, PSO was introduced as global optimizer over continuous spaces, while a large set of POM problems are of combinatorial nature with discrete decision variables" \cite{and}.  In short, he is declaring that the large scale nature of such production problems are indeed appropriate for PSO algorithms.  
While this problem is not large enough to necessitate utilizing such algorithms (a global optimal solution can be determined through linear programming), these implementations are interesting approaches.
\par
Generally speaking, many optimization problems can be converted into a linear programming model.  The drawback to this, however, can be extremely long and computationally expensive runtimes, depending on the solution space.  Klotz and Newman conducted research into this, and 
while they concluded that many operations management/production problems can be solved using linear programming, it may not be the best approach.  However, by adjusting settings and parameters in software-based solvers, many large-scale problems can be effectively solved in this way \cite{klotz}. 
\section{Description of Problem}
Golf-Sport is a golf club manufacturing company with three plants in Arizona.  Each plant has an attached retail store and is treated as its
only supplier.  Each location has predicted minimum and maximum sales of each item type per period.  The goal of this optimization is to maximize profit over a two-month period by determining the optimal number of each product to make
at each location.  In this problem, two types of constraints exist:  local \& global.  Local constraints are unique to each plant/store
 combination, while the global constraints refer to company-wide resources.  Initially, the problem will be modeled and optimized adhering stictly to the provided constraints and bounds.
\section{Mathematical Formulation \& Model}
This problem is set up as a mixed integer programming problem, with a single objective function to be maximized (representing profit).  
The decision variables are set to be the production of each product item, indexed by location and period.  In determining the maximum profit given the constraints, 
the solution will also yield the optimal production plan for Golf-Sport.
\subsection{Assumptions}
Two main assumptions were made in order to create an effective model for this problem.  First, this model represents the company's production as if it were not a 
going-concern.  The model is constructed to optimize profit over two months, thereby not building up any additional inventory for carryover into 
a third month.  By default, the model will not have any inventory left at the end of the second month.  If further information were to arise along with 
a modeling need for a third month, this could be accomplished by expanding the current model. \par
The second assumption used is the carryover of graphite stock between months.  Advertising and graphite allotments are companywide in this problem, and 
it is clearly stated that the amount of money earmarked for advertising each month will not carryover into the next month.  Graphite seems to have no such restriction, so 
it appears reasonable to incorporate the possibility of carryover of graphite stock.
\subsection{Variables \& Indices}
\begin{tabular}{ |c | c| }
\hline
\textbf{Variable} & \textbf{Description} \\
\hline
\textit{x} & Production of Product \textit{i} in Period \textit{j} \\
\textit{e} & Graphite Stock Carryover \\
\hline
\end{tabular}
\vspace{5mm}

\noindent
\begin{tabular}{| c | c |}
\hline
\textbf{Index \textit{i}} & \textbf{Product} \\
\hline
\textit{s} & Steel Shafts \\
\textit{g} & Graphite Shafts \\
\textit{r} & Forged Iron Heads \\
\textit{w} & Metal Wood Heads \\
\textit{h} & Titanium Insert Wood Heads \\
\textit{v} & Set, Steel Shafts, Metal Heads \\
\textit{u} & Set, Steel Shafts, Insert Heads \\
\textit{y} & Set, Graphite Shafts, Metal Heads \\
\textit{z} & Set, Graphite Shafts, Insert Heads \\
\hline
\end{tabular}
\vspace{5mm}
\\
\begin{tabular}{ | c | c | }
\hline
\textbf{Index \textit{j}} & \textbf{Period Description} \\
\hline
1 & Produced \& Sold in Period 1 \\
2 & Produced \& Sold in Period 2 \\
3 & Produced in Period 1 \& Sold in Period 2 \\
\hline
\end{tabular}

\subsection{Objective}
The objective of this model is to maximize profit, represented as the difference between sales revenue and cost.  For the purposes of this function let \textit{R, C \& I} represent revenue, product cost, and inventory cost, respectively.  The function can be expressed as:
%align functions to the left

$$f(x) = \sum_{j=1}^{3}Rx_{i} - \sum_{j=1}^{3}Cx_{i} - \sum Ix_{i,3} $$

The total revenue from all products produced and sold, less the cost of manufacture and inventory, is represented in this function.  While designed for a two-period optimization, the base function can be easily expanded to accomodate for more periods of production and sales.
%superscript for plant, sub for product and period??
\subsection{Constraints}
In construction of this model, the following constraints were constructed within the parameters set forth by the problem:
\begin{equation}
\sum c_{i}x_{i,1} + \sum c_{i}x_{i,3} \leq L_{1} %Labor
\end{equation}
\begin{equation}
\sum c_{i}x_{i,2} \leq L_{2} % Labor
\end{equation}
\begin{equation}
\sum c_{i}x_{i,1} + \sum c_{i}x_{i,3} \leq P_{1} %packing
\end{equation}
\begin{equation}
\sum c_{i}x_{i,2} \leq P_{2}%packing
\end{equation}
\begin{equation}
\sum c_{i}x_{i,1} + \sum c_{i}x_{i,3} \leq A %advertise
\end{equation}
\begin{equation}
\sum c_{i}x_{i,2} \leq A %advertise
\end{equation}
\begin{equation}
\sum c_{i}x_{i,1} + \sum c_{i}x_{i,3} + e \leq G_{1},  i = g, y, z %graphite
\end{equation}
\begin{equation}
\sum c_{i}x_{i,2} \leq G_{2} + e, i = g, y, z %graphite
\end{equation}
\begin{equation}
\sum c_{i}x_{i,1} + \sum c_{i}x_{i,3} \leq U_{1} , i = v, u, y, z %set assembly
\end{equation}
\begin{equation}
\sum c_{i}x_{i,2} \leq U_{2},  i = v, u, y, z %set assembly
\end{equation}
\begin{equation}
Min_{1} \leq x_{i,1} \leq Max_{1} %What is a good variable for the constants for sales bounds
\end{equation}
\begin{equation}
Min_{2} \leq x_{i,2} + x_{i,3} \leq Max_{2}
\end{equation}
\begin{equation}
x_{i,j} \geq 0, integer %how to best put this into math speak
\end{equation}
\begin{equation}
e \geq 0
\end{equation}

Constraints 1-4 represent the labor and packing limitations by plant with cost by product.  The first month's costs include the items manufactured and sold that month, as well as the items manufactured 
for carryover into the second month.  Constraints 5-8 represent the global constraints in this problem:  advertising budget and graphite stock.  The advertising budget is a fixed amount for each month, with no carryover. The 
graphite stock has a consistent amount per month as well, but excess from the first month can be carried over into the next month.  This carryover amount is represented by \textit{e}.  Constraints 9 \& 10 represent the unique assembly time 
for the sets of clubs, again varying by plant location. 
\par
The final constraints (11-14) represent variable bounds.  Constraint 11 ensures that all items produced and sold in period 1 are within the minimum and maximum demand.  Similarly, inequality 12 ensures that the sum of items produced in period 2 and items carryed over to 
period 2 are within that month's demand bounds.  Constraints 13 \& 14 further bound all items to be non-negative.


\section{Implementation}
\subsection{Hardware \& Software}
The problem was coded and solved in MATLAB R2017b on a personal computer.  The code does not appear to be backward-compatible due to the use of the \textit{optimproblem} implementation.  The small scale of this optimization does not necessitate a computer beyond average specifications.
\subsection{Algorithm}
This problem was solved using the \textit{intlinprog} function in MATLAB.  This function conducts a linear relaxation on the problem, in order to set the upper bound.  Then, the algorithm iterates through a branch-and-cut approach to solve for the optimal integer solution.  By default, the function 
will iterate through 1,000,000 iterations prior to terminating.  These iterations check the feasible solutions at the nodes along the tree created through branch-and-bound.  Additionally, the function applies basic heuristics in the form of Gomory cuts.  Gomory cuts are defined as an added constraint along a hyperplane to a vertex.  They are commonly implemented in most commercial integer programming solvers as an initial heuristic.  In optimization software implementations, these cuts tend to be accurate overall and invalid cuts are rarely made \cite{gomory}.  Given this knowledge, it is safe to assume that these cuts will not negatively affect the solution to this problem.

\section{Solution}
The model was successfully optimized to the same solution over several runs, yielding the following optimal solution with constraints as laid out in the original problem above:
\\
\noindent
%begin{table}
\begin{tabular}{ l c c c }
\hline
\multicolumn{4}{|c|}{Optimal Chandler Production} \\
\hline
\textbf{Item} & \textbf{Period 1} & \textbf{Period 2} & \textbf{Period 1 Carryover} \\
Steel Shafts & 0 & 0 & 251 \\
Graphite Shafts & 100 & 2000 & 0 \\
Iron Heads & 200 & 0 & 200 \\
Metal Wood Heads & 35 & 1 & 30 \\
Titanium Insert Wood Heads & 2000 & 1299 & 701 \\
Set, Steel, Metal Heads & 0 & 0 & 0 \\
Set, Steel, Titanium Heads & 0 & 0 & 0 \\
Set, Graphite, Metal Heads & 0 & 0 & 0 \\
Set, Graphite, Titanium Heads & 0 & 35 & 84\\
\end{tabular}
%end{table} dont remove
\vspace{5mm}
\\
\noindent
%begin{table}
\begin{tabular}{ l c c c }
\hline
\multicolumn{4}{|c|}{Optimal Glendale Production} \\
\hline
\textbf{Item} & \textbf{Period 1} & \textbf{Period 2} & \textbf{Period 1 Carryover} \\
Steel Shafts & 0 & 0 & 0 \\
Graphite Shafts & 100 & 100 & 0 \\
Iron Heads & 200 & 199 & 1 \\
Metal Wood Heads & 36 & 30 & 6 \\
Titanium Insert Wood Heads & 2000 & 1289 & 711 \\
Set, Steel, Metal Heads & 0 & 0 & 0 \\
Set, Steel, Titanium Heads & 0 & 0 & 0 \\
Set, Graphite, Metal Heads & 0 & 0 & 0 \\
Set, Graphite, Titanium Heads & 0 & 68 & 83\\
\end{tabular}
%end{table} dont remove
\vspace{5mm}
\\
\noindent
%begin{table}
\begin{tabular}{ l c c c }
\hline
\multicolumn{4}{|c|}{Optimal Tucson Production} \\
\hline
\textbf{Item} & \textbf{Period 1} & \textbf{Period 2} & \textbf{Period 1 Carryover} \\
Steel Shafts & 1 & 0 & 2 \\
Graphite Shafts & 431 & 1265 & 2 \\
Iron Heads & 100 & 0 & 100 \\
Metal Wood Heads & 339 & 0 & 139 \\
Titanium Insert Wood Heads & 2000 & 814 & 1186 \\
Set, Steel, Metal Heads & 0 & 0 & 0 \\
Set, Steel, Titanium Heads & 0 & 0 & 0 \\
Set, Graphite, Metal Heads & 0 & 0 & 0 \\
Set, Graphite, Titanium Heads & 0 & 92 & 92\\
\end{tabular}
%end{table} dont remove
\\
\\
\vspace{5mm}
\\

This optimal production scenario generates a maximized profit of \$940,366 for Golf-Sport as a company in the two month period.  The linear relaxation yielded a global optimal profit of \$941,110, so this 
IP solution is valid (does not violate the Representation Theorem).  The problem does not provide a current company profit, so it is not known how much of an increase in potential profit this soution would yield.  

\section{Further Analysis}
The initial model and code provides a valid, optimal solution.  While the solution is valid, there are several other points to be addressed to create a more meaningful model.  Hence, the problem will be expanded upon further in two separate directions.
\par
First, while the original analysis and model provides a valid optimal integer solution for the profit function, the actual production plan does not appear very practical for a business.  
The existing parameters seem to not account for maintaining a workforce; the lower bounds are set too low.  This analysis will adjust some of these parameters and constraints to produce a more 
actionable business plan.
\par
Second, the problem will be analyzed and adjusted to explore several range sensitivity topics.  This can help direct the business' focus and direction in order to further maximize profit potential.  Specifically, advertising dollars and graphite stock range sensitivity will be explored.  These are both areas in which Golf-Sport has considered contributing extra resources.  Additionally, a proposed advertising plan 
that could potentially double the maximum demand will be explored within current model parameters.
\section{Added Practicality Constraints}
Add in more practical minimums to develop a more realistic solution %needs finished
\section{Range Sensitivity}
Range sensitivity and resource increases were examined in regards to this problem.  The two main points of focus (as supplied by the case study) are a hypothetical increase in total maximum demand and proposed increases in advertising or graphite budget.  Both of these situations 
will be examined independently, using the initial mathematical model as the basis.
\subsection{Advertising Budget v. Graphite Supply}
The case study of the Golf-Sport problem presents a situation:  which would be more beneficial for the company, more monthly advertising budget or more graphite stock?  Since an amount of increase was not given, the situation was modeled with double the resources in the original problem.  
This allowed for a \$40,000 monthly advertising budget or a 2,000 pound monthly graphite stock.  As was the case before, advertising budget did not carry over to the next period, but graphite stock was assumed to.
\par
These new parameters were incorporated separately into two different runs of the model.  Graphite was determined to be the more valuable resource, as it increases the maximum profit to \$968,831.  This profit potential is notably larger than the baseline of \$940,366 and the increased advertising 
profit of \$940,110.  This should help show Golf-Sport the importance of graphite to the operation, as it is a limiting factor on current profit.  It should be noted that when graphite is doubled in this case, excess stock exists at the end of the two-month period.  This suggests that with current demand and constraints, obtaining more than 2,000 pounds of graphite stock per month is unnecessary, 
as it will not increase profit beyond this point.
\subsection{Doubling the Demand}
The Golf-Sport case study also proposes an interesting change:  if maximum demand were to double, would the company be able to benenfit and maximize their profit while utilizing the current production system as originally described?  The model was adjusted to account for a hypoethetical doubling of maximum demand while maintaining all of the original constraints as set forth in the model.  
Using this modified model, an integer solution was found.  The objective value (maximum profit) is optimized at \$998,330.  This is approximately a \$48,000 greater profit than the original demands yielded.  This shows that the model does benefit from the increased demand.
\par
Here to discuss slack %discuss slack and see which constraints are maxed out
\section{Conclusions \& Implications}
\subsection{Mathematical Approach}
Evaluate our approach, suggest changes in future.
\subsection{Company Approach}
Discuss optimized solution, range sensitivity, business suggestions



	\newpage
	\pagenumbering{gobble}
	\begin{thebibliography}{5}
	
\bibitem{survey}
Fei, Hongying, et al. (2017). A Survey of Recent Research on Optimization Models and Algorithms for Operations Management from the Process View. 
\textit{Scientific Programming, 2017}(2017), pp. 1-19.
	
\bibitem{gen}
Aytug, H. et al. (2003). Use of Genetic Algorithms to Solve Production and Operations Management Problems: A Review. 
\textit{International Journal of Production Research, 41}(17), pp. 3955-4009.

\bibitem{and}
Nearchou, Andreas C. (2011). Maximizing Production Rate and Workload Smoothing in Assembly Lines Using Particle Swarm Optimization. 
\textit{International Journal of Production Economics, 129}(2), pp. 242-250.

\bibitem{klotz}
Klotz, Ed, and Alexandra M. Newman. (2013). Practical Guidelines for Solving Difficult Linear Programs. 
\textit{Practical Guidelines for Solving Difficult Linear Programs, 18}(1-2), pp. 1-17.

\bibitem{gomory}
G\'{e}rard Cornu\'{e}jols, Fran\c{c}ois Margot, \& Giacomo Nannicini. (2013). On the Safety of Gomory Cut Generators. 
\textit{Mathematical Programming Computation, 5}(4), pp. 345-395.


\end{thebibliography}
\end{document}
