\documentclass{article}
\title{Golf-Sport:  Managing Operations \\ A Case Study in Optimization}
\date{May 8, 2018}
\author{John D. Bulger \& Matthew T. Hess, Valparaiso University}
\usepackage[margin=1.5in]{geometry}
\begin{document}
	\pagenumbering{gobble}
	\maketitle
	\newpage
	\pagenumbering{arabic}
\section{Introduction \& Background}
\subsection{Operations Research \& Optimization}
Optimizing profit through production in processes is studied within the realm of operations management. As the global market continues to grow and expand, businesses are focusing on remaining competitive. Erik Johannesson and Christian Karsten explain this best. ``Large organizations face thousands of planning problems every day. Some are simple, and decisions can be made quickly and efficiently by a single person. Others are on a different, larger scale which require much more effort, and they may also have a significant impact on the bottom line.  All of these decisions are interlinked, with the overall objective being to meet customer demand at the lowest overall cost."\cite{johan}
\par
In a survey of optimization techniques being used within operations management, the authors have broken the realm into three distinct sectors. These can be identified as input management, facilities management, and output management. Stated more simply, these correspond to scheduling, production, and vehicle routing, respectively. Failure to optimize any one of these areas can to lead large inefficiencies (i.e. lost revenue) for a company. Making too much of one product leads to inventory costs, failure to plan good routes leads to higher shipping costs, and not spending enough on advertising leads to missed opportunities and subpar performance. Johnannesson and Karsten go on to say, ``As such problems often have a large scale, the impact can be significant. For example, it is not uncommon to be able to decrease logistics spending by 5 to 10 percent in a supply chain with annual logistics costs of over \$100 million. 
Additionally, it can reduce the planning time from days to minutes. Thus, evaluation of multiple scenarios becomes feasible, and an automated decision-support tool may serve as an integral part of both operational and strategic decision making."\cite{johan}  
\subsection{Prior Work}
There is a clear need for operations management to be optimized, especially as a business or problem grows more complex. Several approaches for operations management use genetic algorithms, especially when focusing on lot sizing within material requirements, aggregate and assortment planning, and facility layout problems.\cite{gen}  
Additionally, particle swarm optimization methods have been successfully implemented into operations research problems.  Nearchou expands on this by stating that ``A possible reason for this absence is that, PSO was introduced as global optimizer over continuous spaces, while a large set of POM problems are of combinatorial nature with discrete decision variables."\cite{and}  In short, he is declaring that the large scale nature of such production problems are indeed appropriate for PSO algorithms.  
While this problem is not large enough to necessitate utilizing such algorithms (a global optimal solution can be determined through linear programming), these implementations are interesting approaches.
\par
Generally speaking, many optimization problems can be converted into a linear programming model.  The drawback to this, however, can be extremely long and have computationally expensive runtimes, depending on the solution space.  Klotz and Newman conducted research into this, and 
while they concluded that many operations management/production problems can be solved using linear programming, it may not always be the best approach.  However, by adjusting settings and parameters in software-based solvers, many large-scale problems can be effectively solved in this way.\cite{klotz}  This Golf-Sport problem falls squarely within the facilities management/production sector.  Fei et al. discovered that for large-scale production optimizations, heuristics were necessarily utilized.  Additionally, the most common solution approaches utilize branch-and-cut or branch-and-price.\cite{survey}
\section{Description of Problem}
Golf-Sport is a small-sized company that produces high-quality components for people who build their own golf clubs and prebuilt sets of clubs. There are five components:  steel shafts, graphite shafts, forged iron heads, metal wood heads, and metal wood heads with titanium inserts; these products are produced in three plants:  Chandler, Glendale, and Tucson. Each plant can produce any of the components, although each plant has a different set of individual constraints and unit costs. These constraints cover labor and packaging machine time (the machine is used for all components). Note that even though the components are identical in the three plants, different production processes are used, and therefore the products use different amounts of resources in different plants.
\par
Besides component sales, the company takes the components and manufactures sets of golf clubs. Each set requires 13 shafts, 10 iron heads, and 3 wood heads. All of the shafts in a set must be the same type (steel or graphite), and all of the wood heads must be the same type (metal or metal with inserts). Each plant has unique assembly times.  Furthermore, each plant of Golf-Sport has a retail outlet to sell components and sets, and the specific plant is the only supplier for its retail outlet. The minimum and maximum amounts of demand for each location are unique.
\par
This planning problem is for two months. The material, production, and assembly costs increase by 12\% for the second month, and production times are stationary. Inventory costs are based on end-of-period inventory for each product set and cost out at 8\% of the cost values of material, production and assembly. Additionally, each product generates different revenue depending on where it is made. Initially, there is no inventory.
\par
The corporation controls the capital available or expenses; the cash requirements for each product are given. There is a total of \$20,000 available for advertising for the entire system during each month, and any money not spent in a month is not available the next month. The corporation also controls graphite. Each shaft requires 4 ounces of graphite; a total of 1,000 pounds is available for each of the two months. 
\par
The goal of this optimization is to develop a production plan that will maximize profit for the company. Additionally, the following sensitivity-analysis issues will be explored:
\begin{itemize}
	\item How can the problem be modeled to present a more complete picture of a manufacturer’s operations?
	\item How does more advertising budget or graphite affect the outcome?
	\item Marketing is trying to get Golf-Sport to consider an advertising program that promises a 50\% increase in their maximum demand. Can the current system be utilized to increase profit with this new scheme?
\end{itemize}
\par
The solution will presented as items produced and sold according to the following method: (1) manufactured and sold in month one; (2) manufactured and sold in month two; and (3) manufactured in month one and sold in month two. In this problem, two types of constraints exist: local and global. This two-month optimization can serve as a baseline and could later be expanded to more periods if desired. This particular problem appears to lend itself to integer programming as the ideal optimization method. The problem will be coded and solved using MATLAB's optimization toolbox functionality.
\section{Mathematical Formulation \& Model}
This problem is set up as a mixed integer programming problem, with a single objective function to be maximized (representing profit).  
The decision variables are set to be the production of each product item, indexed by location and period.  In determining the maximum profit given the constraints, 
the solution will also yield the optimal production plan for Golf-Sport.
\subsection{Assumptions}
Two main assumptions were made in order to create an effective model for this problem.  First, this model represents the company's production as if it were not a 
going-concern.  The model is constructed to optimize profit over two months, thereby not building up any additional inventory for carryover into 
a third month.  By default, the model will not have any inventory left at the end of the second month.  If further information were to arise along with 
a modeling need for a third month, this could be accomplished by expanding the current model.
\par
The second assumption concerns the carryover of graphite stock between months.  Advertising and graphite allotments are companywide in this problem, and 
it is clearly stated that unused money earmarked for advertising each month will not carry over into the next month.  Graphite has no such restriction, so 
it seems reasonable to incorporate the possibility of carryover of graphite stock.
\subsection{Variables \& Indices}

\begin{tabular}{ |c | c| }
\hline
\textbf{Variable} & \textbf{Description} \\
\hline
\textit{x} & Production/Sale of Product \textit{i} in Period \textit{j} \\
\textit{e} & Graphite Stock Carryover \\
\hline
\end{tabular}
\vspace{5mm}

\noindent
\begin{tabular}{| c | c |}
\hline
\textbf{Index \textit{i}} & \textbf{Product} \\
\hline
\textit{s} & Steel Shafts \\
\textit{g} & Graphite Shafts \\
\textit{r} & Forged Iron Heads \\
\textit{w} & Metal Wood Heads \\
\textit{h} & Titanium Insert Wood Heads \\
\textit{v} & Set, Steel Shafts, Metal Heads \\
\textit{u} & Set, Steel Shafts, Insert Heads \\
\textit{y} & Set, Graphite Shafts, Metal Heads \\
\textit{z} & Set, Graphite Shafts, Insert Heads \\
\hline
\end{tabular}
\vspace{5mm}
\\
\begin{tabular}{ | c | c | }
\hline
\textbf{Index \textit{j}} & \textbf{Period Description} \\
\hline
1 & Produced \& Sold in Period 1 \\
2 & Produced \& Sold in Period 2 \\
3 & Produced in Period 1 \& Sold in Period 2 \\
\hline
\end{tabular}

\subsection{Objective}
The objective of this model is to maximize profit, represented as the difference between sales revenue and cost.  For the purposes of this function let \textit{R, C,} and \textit{I} represent revenue, product cost, and inventory cost, respectively.  The function can be expressed as:
%align functions to the left

$$f(x) = \sum{Rx_{i,j}} - \sum{Cx_{i,j}} - \sum Ix_{i,3} $$

The total revenue from all products produced and sold, minus the cost of manufacture and inventory, is represented in this function.  While designed for a two-period optimization, the base function can be easily expanded to accomodate for more periods of production and sales.
%superscript for plant, sub for product and period??
\subsection{Constraints}
In construction of this model, the following constraints were constructed within the parameters set forth by the problem:
\begin{equation}
\sum c_{i}x_{i,1} + \sum c_{i}x_{i,3} \leq L_{1} %Labor
\end{equation}
\begin{equation}
\sum c_{i}x_{i,2} \leq L_{2} % Labor
\end{equation}
\begin{equation}
\sum c_{i}x_{i,1} + \sum c_{i}x_{i,3} \leq P_{1} %packing
\end{equation}
\begin{equation}
\sum c_{i}x_{i,2} \leq P_{2}%packing
\end{equation}
\begin{equation}
\sum c_{i}x_{i,1} + \sum c_{i}x_{i,3} \leq A %advertise
\end{equation}
\begin{equation}
\sum c_{i}x_{i,2} \leq A %advertise
\end{equation}
\begin{equation}
\sum c_{i}x_{i,1} + \sum c_{i}x_{i,3} + e \leq G_{1},  i = g, y, z %graphite
\end{equation}
\begin{equation}
\sum c_{i}x_{i,2} \leq G_{2} + e, i = g, y, z %graphite
\end{equation}
\begin{equation}
\sum c_{i}x_{i,1} + \sum c_{i}x_{i,3} \leq U_{1} , i = v, u, y, z %set assembly
\end{equation}
\begin{equation}
\sum c_{i}x_{i,2} \leq U_{2},  i = v, u, y, z %set assembly
\end{equation}
\begin{equation}
Min_{1} \leq x_{i,1} \leq Max_{1} %What is a good variable for the constants for sales bounds
\end{equation}
\begin{equation}
Min_{2} \leq x_{i,2} + x_{i,3} \leq Max_{2}
\end{equation}
\begin{equation}
x_{i,j} \geq 0, integer %how to best put this into math speak
\end{equation}
\begin{equation}
e \geq 0
\end{equation}

Constraints 1-4 represent the labor and packing limitations by plant with cost by product.  The first month's costs include the items manufactured and sold that month, as well as the items manufactured 
for carryover into the second month.  Constraints 5-8 represent the global constraints in this problem:  advertising budget and graphite stock.  The advertising budget is a fixed amount for each month, with no carryover. The 
graphite stock has a consistent amount per month as well, but excess from the first month can be carried over into the next month.  This carryover amount is represented by \textit{e}.  Constraints 9 and 10 represent the unique assembly time 
for the sets of clubs, again varying by plant location. 
\par
The final constraints (11-14) represent variable bounds.  Constraint 11 ensures that all items produced and sold in period 1 are within the minimum and maximum demand.  Similarly, inequality 12 ensures that the sum of items produced in period 2 and items carried over to 
period 2 are within that month's demand bounds.  Constraints 13 and 14 further bound all items to be non-negative.


\section{Implementation}
\subsection{Hardware \& Software}
The problem was coded and solved in MATLAB R2017b on a personal computer.  The code does not appear to be backward-compatible due to the use of the \textit{optimproblem} implementation.  The small scale of this optimization does not necessitate a computer beyond average specifications.
\subsection{Algorithm}
This problem was solved using the \textit{intlinprog} function in MATLAB.  This function initially conducts a linear relaxation on the problem, in order to set the upper bound.  Then, the algorithm iterates through a branch-and-cut approach to solve for the optimal integer solution.  By default, the function 
will iterate through 1,000,000 iterations prior to terminating or until the gap tolerance is met.  These iterations check the feasible solutions at the nodes along the tree created through branch-and-bound.  Additionally, the function applies basic heuristics in the form of Gomory cuts.  Gomory cuts are defined as an added constraint along a hyperplane to a vertex.  They are commonly implemented in most commercial integer programming solvers as an initial heuristic.  In optimization software implementations, these cuts tend to be accurate overall and invalid cuts are rarely made.\cite{gomory}  Given this knowledge, it is safe to assume that these cuts will not negatively affect the solution to this problem.

\section{Solution}
The model was successfully optimized to the same solution over several runs, yielding the following optimal solution with constraints as laid out in the original problem above:
\\
\\
\\
\noindent
%begin{table}
\begin{tabular}{ l | c | c |c }
\hline
\multicolumn{4}{|c|}{Optimal Chandler Production} \\
\hline
\textbf{Item} & \textbf{Period 1} & \textbf{Period 2} & \textbf{Period 1 Carryover} \\
Steel Shafts & 0 & 0 & 251 \\
Graphite Shafts & 100 & 2000 & 0 \\
Iron Heads & 200 & 0 & 200 \\
Metal Wood Heads & 35 & 1 & 30 \\
Titanium Insert Wood Heads & 2000 & 1299 & 701 \\
Set, Steel, Metal Heads & 0 & 0 & 0 \\
Set, Steel, Titanium Heads & 0 & 0 & 0 \\
Set, Graphite, Metal Heads & 0 & 0 & 0 \\
Set, Graphite, Titanium Heads & 0 & 35 & 84\\
\end{tabular}
%end{table} dont remove
\vspace{5mm}
\\
\noindent
%begin{table}
\begin{tabular}{ l | c | c | c }
\hline
\multicolumn{4}{|c|}{Optimal Glendale Production} \\
\hline
\textbf{Item} & \textbf{Period 1} & \textbf{Period 2} & \textbf{Period 1 Carryover} \\
Steel Shafts & 0 & 0 & 0 \\
Graphite Shafts & 100 & 100 & 0 \\
Iron Heads & 200 & 199 & 1 \\
Metal Wood Heads & 36 & 30 & 6 \\
Titanium Insert Wood Heads & 2000 & 1289 & 711 \\
Set, Steel, Metal Heads & 0 & 0 & 0 \\
Set, Steel, Titanium Heads & 0 & 0 & 0 \\
Set, Graphite, Metal Heads & 0 & 0 & 0 \\
Set, Graphite, Titanium Heads & 0 & 68 & 83\\
\end{tabular}
%end{table} dont remove
\vspace{5mm}
\\
\noindent
%begin{table}
\begin{tabular}{ l | c | c | c }
\hline
\multicolumn{4}{|c|}{Optimal Tucson Production} \\
\hline
\textbf{Item} & \textbf{Period 1} & \textbf{Period 2} & \textbf{Period 1 Carryover} \\
Steel Shafts & 1 & 0 & 2 \\
Graphite Shafts & 431 & 1265 & 2 \\
Iron Heads & 100 & 0 & 100 \\
Metal Wood Heads & 339 & 0 & 139 \\
Titanium Insert Wood Heads & 2000 & 814 & 1186 \\
Set, Steel, Metal Heads & 0 & 0 & 0 \\
Set, Steel, Titanium Heads & 0 & 0 & 0 \\
Set, Graphite, Metal Heads & 0 & 0 & 0 \\
Set, Graphite, Titanium Heads & 0 & 92 & 92\\
\end{tabular}
%end{table} dont remove
\\
\\
\vspace{5mm}
\\

This optimal production scenario generates a maximized profit of \$940,366 for Golf-Sport as a company in the two month period.  The linear relaxation yielded a global optimal profit of \$941,110, so this 
IP solution is valid (because it does not violate the Representation Theorem).  The problem does not provide a current company profit prior to this optimization.  Therefore, it is not known how this production scheme compares to the company's current production and profit.

\section{Further Analysis}
The initial model and code provides a valid, optimal solution.  While the solution is valid, there are several other points to be addressed to create a more meaningful model.  Hence, the problem will be expanded upon further in two separate directions.
\par
First, while the original analysis and model does maximize profit, the actual production plan does not appear completely practical for a business recommendation.  For example, the existing parameters seem to not account for maintaining a workforce, and the minimum demand is set too low to stock the shelves of a retail outlet.  This analysis will adjust some of these parameters and constraints to produce a more 
actionable business plan.
\par
Secondly, the problem will be analyzed and adjusted to explore several range sensitivity topics.  This can help direct the business' focus and direction in order to further maximize profit potential.  Specifically, advertising dollars and graphite stock range sensitivity will be explored.  These are both areas in which Golf-Sport has considered contributing extra resources.  Additionally, a proposed advertising plan 
that could potentially double the maximum demand will be explored within current model parameters.
\section{Added Practicality Constraints}
\subsection{Modifications to Existing Model}
Upon further examination and analysis of the model \& results, the current parameters do not appear to be as complete as possible when representing a manufacturing company.  While it exists only as a case study, several further parameters will be added in order to present a more realistic model and solution.  The added parameters will include:
\begin{itemize}
	\item Addition of minimum labor constraints
	\item Addition of basic start-up cost constraints
	\item Increasing of minimum demand for items with a current production minimum of zero
	\item Marginal increase in per-item revenue for period two to help offset for cost increases
	\item Incorporate parameters to reflect items lost in quality control checks
\end{itemize}
The addition of these parameters is necessary to model a more complete production problem.  First, the current model yields very uneven production amounts between periods.  While perhaps optimal, it would be unrealistic to recommend that a company lay off its workers if unnecessary.  Therefore, labor minimums will be set to 25\% of the current maximums.
\par
Second, any production facility will incur fixed costs per month, regardless of production.  This will be included as a constraint:
$$ mq_{j} >= \sum_{i} x_{j}, q\in {0,1} $$
This start-up operating cost will be modeled as \$7,500 per plant per period.  While the model will include this as a binary equation, it will effectively always be included due to the minimum labor constraints.  The addition of the binary constraint, however, allows for ease of possible analysis without minimum production.  The constant \textit{m} will be set to an arbitrarily high number, so as to not limit any production in error.  Once this constraint is added to the model, the objective function will change to reflect it, with \textit{S} representing start-up cost:
$$f(x) = \sum_{j=1}^{3}Rx_{i} - \sum_{j=1}^{3}Cx_{i} - \sum Ix_{i,3} -\sum Sq_{j}$$
\par
Next, the minimum for items with a current minimum demand of zero will be increased to 5 of each item.  For retail store purposes, even items with underwhelming sales should be stocked on the store shelves.  By increasing this minimum demand slightly, it will ensure that each location produces a full catalog of items.  
\par
Additionally, the increase in cost and inventory carryover cost is not currently reflected in item revenue for the second month.  As costs increases significantly, at least some of that cost should be passed along to the customer.  Therefore, the modifed model will include a 5\% revenue (price) increase per item sold in the second month.
\par
Lastly, in any production setting, items are lost to defects and quality control audits.  Six Sigma production guidelines set a goal of 3.4 defects per million items.  For the purposes of this model, it will be assumed that Golf-Sport is far from the Six Sigma efficiency levels.  To reflect this, Golf-Sports total revenue formula will be adjusted to 99.5\% to account for items lost due to defects.
\subsection{Modifed Solution}
After factoring in these additional constraints, the problem was optimized to yield \$833,400 in profit from the following production scheme:
\\
\\
\\
\noindent
%begin{table}
\begin{tabular}{ l | c | c | c }
\hline
\multicolumn{4}{|c|}{Optimal Modified Chandler Production} \\
\hline
\textbf{Item} & \textbf{Period 1} & \textbf{Period 2} & \textbf{Period 1 Carryover} \\
Steel Shafts & 5 & 5 & 5 \\
Graphite Shafts & 100 & 1,490 & 5 \\
Iron Heads & 200 & 194 & 6 \\
Metal Wood Heads & 30 & 25 & 5 \\
Titanium Insert Wood Heads & 2000 & 1750 & 250 \\
Set, Steel, Metal Heads & 5 & 5 & 5 \\
Set, Steel, Titanium Heads & 5 & 5 & 5 \\
Set, Graphite, Metal Heads & 5 & 5 & 5 \\
Set, Graphite, Titanium Heads & 5 & 5 & 49\\
\end{tabular}
%end{table} dont remove
\vspace{5mm}
\\
\noindent
%begin{table}
\begin{tabular}{ l | c | c | c }
\hline
\multicolumn{4}{|c|}{Optimal Modified Glendale Production} \\
\hline
\textbf{Item} & \textbf{Period 1} & \textbf{Period 2} & \textbf{Period 1 Carryover} \\
Steel Shafts & 5 & 5 & 5 \\
Graphite Shafts & 101 & 549 & 6 \\
Iron Heads & 200 & 195 & 5 \\
Metal Wood Heads & 30 & 25 & 5 \\
Titanium Insert Wood Heads & 1859 & 1995 & 5 \\
Set, Steel, Metal Heads & 5 & 5 & 5 \\
Set, Steel, Titanium Heads & 5 & 5 & 5 \\
Set, Graphite, Metal Heads & 5 & 5 & 5 \\
Set, Graphite, Titanium Heads & 5 & 5 & 48\\
\end{tabular}
%end{table} dont remove
\vspace{5mm}
\\
\noindent
%begin{table}
\begin{tabular}{ l | c | c | c }
\hline
\multicolumn{4}{|c|}{Optimal Modified Tucson Production} \\
\hline
\textbf{Item} & \textbf{Period 1} & \textbf{Period 2} & \textbf{Period 1 Carryover} \\
Steel Shafts & 5 & 5 & 5 \\
Graphite Shafts & 1,117 & 1,919 & 81 \\
Iron Heads & 100 & 94 & 6 \\
Metal Wood Heads & 15 & 8 & 7 \\
Titanium Insert Wood Heads & 2000 & 1,940 & 60 \\
Set, Steel, Metal Heads & 5 & 5 & 5 \\
Set, Steel, Titanium Heads & 5 & 5 & 5 \\
Set, Graphite, Metal Heads & 5 & 5 & 5 \\
Set, Graphite, Titanium Heads & 5 & 5 & 57\\
\end{tabular}
%end{table} dont remove
\\
\\
\vspace{5mm}
\\
As can be seen, this is a much more balanced production plan that still serves to optimize the company's profit.  Production is more consistent thoroughout periods, keeping the workforce employed and the store shelves stocked.  While this optimized profit is less than the original model, it does allow for a more realistic view of Golf-Sport's operations.
\section{Range Analysis}
Range sensitivity and resource increases were examined in regard to this problem.  The two main points of focus are a hypothetical increase in total maximum demand and proposed increases in advertising or graphite budget.  Both of these situations 
will be examined independently, using the initial mathematical model (not the modified model from the prior section) as the basis.
\subsection{Advertising Budget v. Graphite Supply}
The case study of the Golf-Sport problem presents a situation:  which would be more beneficial for the company, more monthly advertising budget or more graphite stock?  Since an amount of increase was not given, the situation was modeled with double the resources in the original problem.  
This allowed for a \$40,000 monthly advertising budget or a 2,000 pound monthly graphite stock.  As was the case before, advertising budget did not carry over to the next period, but graphite stock was assumed to.
\par
These new parameters were incorporated separately into two different runs of the model.  Graphite was determined to be the more valuable resource, as it increases the maximum profit to \$968,831.  This profit potential is notably larger than the baseline of \$940,366 and the increased advertising 
profit of \$941,110.  This should help show Golf-Sport the importance of graphite to the operation, as it is a limiting factor on current profit.  It should be noted that when graphite is doubled in this case, excess stock exists at the end of the two-month period.  This suggests that with current demand and constraints, obtaining more than 2,000 pounds of graphite stock per month is unnecessary, 
as it will not directly increase profit beyond this point.
\subsection{Doubling the Demand}
The Golf-Sport case study also proposes an interesting change:  if maximum demand were to double, would the company be able to benenfit and maximize their profit while utilizing the current production system as originally described?  The model was adjusted to account for a hypoethetical doubling of maximum demand while maintaining all of the original constraints as set forth in the model.  
Using this modified model, an integer solution was found.  The objective value (maximum profit) is optimized at \$998,330.  This is approximately a \$48,000 greater profit than the original demands yielded.  This shows that the model does benefit from the increased demand.  However, this does not take into account any cost of advertising or promotion to develop this demand.

\section{Conclusions \& Implications}
\subsection{Mathematical Approach}
Mathematically speaking, the Golf-Sport case study was successfully modeled, coded, and solved.  Depending on the setup of the code, each run took between 13 seconds and 30 minutes for completion.  When the problem was modeled as a linear relaxation, solutions took seconds to be produced.  Solutions were effectively produced and passed spot testing on various constraints in an attempt to verify solution integrity.
\par
Critically viewed, the model was effective but perhaps inefficient, especially if extended beyond the two-month period under analysis.  From a retrospective viewpoint, the model could possibly have been developed with the variables as ``produced" and ``sold", eliminating the need for the ambiguous Period 3 connotation.  A model developed in this sense could be more easily expanded to more subsequent periods, increasing its utility to the organization.  It should be noted that these possible adjustments, however, would not produce any different results than what has been achieved in the current model.  An example of such a modifed objective function could be:
$$ f(x) = \sum Rs_{i,j} - \sum Cp_{i,j} - \sum I(p_{i,j}-s_{i,j}) $$
This function sets \textit{s} and \textit{p} as sold and produced items, with the difference between the two per period being the carryover inventory.
\par
In summary, the mathematical formulation and application on the Golf-Sport case study can be viewed as successful and easily modified to evaluate different parameters and constraints.
\subsection{Company Approach}
Upon completion of this analysis, it is evident that Golf-Sport can easily maximize their profit by utilizing such a model.  The parameters included, however, are superficial and unrealistic in many cases.  The problem modification above adjusts some constraints and adds to others, so as to help illustrate the process of developing a more complete picture of Golf-Sport operations.  While the actual numbers in the formulation are either arbitrary or unique to this case study, the modeling method utilized could be expanded and modified to fit many other production situations.  Some possible other questions that can be explored in the future using such a model and case could be:
\begin{itemize}
	\item When does is become feasible to open a new plant or expand an existing one?
	\item How does eliminating certain products (e.g. steel shafts) impact sales? Does it increase profit by shifting the customer base to another product? Can Golf-sport handle the increased demand of the other products?
	\item How does the shifting cost of materials (other than graphite) change what is produced? How does it impact profits?
	\item What do profits look like when evaluating beyond a two-month period?
\end{itemize}
In short, the current model can be expanded on in many different directions depending on the goal of the analysis.  It is by this measure that this optimization approach can be deemed successful, as continuing analysis is of paramount importance in operations management.


	\newpage
	\pagenumbering{gobble}
	\begin{thebibliography}{6}
	
\bibitem{johan}
Johannesson, Erik, and Christian Karsten. (2017, 18 May). Solving Complex Planning Problems Using Old Math: Optimization Applied to Operations Planning [Blog Post]. Retrieved 
from http://https://medium.com/digital-mckinsey/solving-complex-planning-problems-using-old-math-optimization-applied-to-operations-planning-de7f79610064.
	
\bibitem{gen}
Aytug, H. et al. (2003). Use of Genetic Algorithms to Solve Production and Operations Management Problems: A Review. 
\textit{International Journal of Production Research, 41}(17), pp. 3955-4009.

\bibitem{and}
Nearchou, Andreas C. (2011). Maximizing Production Rate and Workload Smoothing in Assembly Lines Using Particle Swarm Optimization. 
\textit{International Journal of Production Economics, 129}(2), pp. 242-250.

\bibitem{klotz}Klotz, Ed, and Alexandra M. Newman. (2013). Practical Guidelines for Solving Difficult Linear Programs. 
\textit{Practical Guidelines for Solving Difficult Linear Programs, 18}(1-2), pp. 1-17.

\bibitem{survey}
Fei, Hongying, et al. (2017). A Survey of Recent Research on Optimization Models and Algorithms for Operations Management from the Process View. 
\textit{Scientific Programming, 2017}(2017), pp. 1-19.

\bibitem{gomory}
G\'{e}rard Cornu\'{e}jols, Fran\c{c}ois Margot, \& Giacomo Nannicini. (2013). On the Safety of Gomory Cut Generators. 
\textit{Mathematical Programming Computation, 5}(4), pp. 345-395.
\\
\\
\\
\\
\\
``We have neither given or received, nor have we tolerated others' use of unauthorized aid."


\end{thebibliography}
\end{document}
